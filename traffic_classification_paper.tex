
\documentclass[12pt]{article}
\usepackage{graphicx}
\usepackage{amsmath}
\usepackage{geometry}
\geometry{a4paper, margin=1in}

\title{Traffic Classification using Convolutional Neural Networks}
\author{Your Name}
\date{\today}

\begin{document}

\maketitle

\begin{abstract}
Traffic classification is a crucial task in network management and security. This paper explores the use of Convolutional Neural Networks (CNNs) to classify traffic patterns. The proposed model achieved high accuracy and efficiency, leveraging GPU acceleration for training.
\end{abstract}

\section{Introduction}
Traffic classification involves analyzing data flow to identify patterns and anomalies. Traditional methods rely on manual feature extraction, but deep learning methods, particularly CNNs, can automate this process.

\section{Methodology}
\subsection{Dataset}
The dataset contains features such as traffic flow statistics and labels indicating the traffic type. Preprocessing steps included normalization and splitting into training and test sets.

\subsection{Model Architecture}
The CNN model consists of a convolutional layer, pooling layer, dropout regularization, and dense layers. The model was trained using the Adam optimizer and categorical crossentropy loss.

\section{Results}
The model achieved a test accuracy of over 90\%, demonstrating its capability to classify traffic patterns effectively.

\section{Conclusion}
This study highlights the effectiveness of CNNs in traffic classification tasks. Future work may focus on real-time classification and handling imbalanced datasets.

\section*{References}
\begin{enumerate}
    \item Reference 1
    \item Reference 2
\end{enumerate}

\end{document}
